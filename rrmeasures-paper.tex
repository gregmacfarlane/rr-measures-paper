% Options for packages loaded elsewhere
% Options for packages loaded elsewhere
\PassOptionsToPackage{unicode}{hyperref}
\PassOptionsToPackage{hyphens}{url}
\PassOptionsToPackage{dvipsnames,svgnames,x11names}{xcolor}
%
\documentclass[
  authoryear,
  review,
  3p]{elsarticle}
\usepackage{xcolor}
\usepackage{amsmath,amssymb}
\setcounter{secnumdepth}{5}
\usepackage{iftex}
\ifPDFTeX
  \usepackage[T1]{fontenc}
  \usepackage[utf8]{inputenc}
  \usepackage{textcomp} % provide euro and other symbols
\else % if luatex or xetex
  \usepackage{unicode-math} % this also loads fontspec
  \defaultfontfeatures{Scale=MatchLowercase}
  \defaultfontfeatures[\rmfamily]{Ligatures=TeX,Scale=1}
\fi
\usepackage{lmodern}
\ifPDFTeX\else
  % xetex/luatex font selection
\fi
% Use upquote if available, for straight quotes in verbatim environments
\IfFileExists{upquote.sty}{\usepackage{upquote}}{}
\IfFileExists{microtype.sty}{% use microtype if available
  \usepackage[]{microtype}
  \UseMicrotypeSet[protrusion]{basicmath} % disable protrusion for tt fonts
}{}
\makeatletter
\@ifundefined{KOMAClassName}{% if non-KOMA class
  \IfFileExists{parskip.sty}{%
    \usepackage{parskip}
  }{% else
    \setlength{\parindent}{0pt}
    \setlength{\parskip}{6pt plus 2pt minus 1pt}}
}{% if KOMA class
  \KOMAoptions{parskip=half}}
\makeatother
% Make \paragraph and \subparagraph free-standing
\makeatletter
\ifx\paragraph\undefined\else
  \let\oldparagraph\paragraph
  \renewcommand{\paragraph}{
    \@ifstar
      \xxxParagraphStar
      \xxxParagraphNoStar
  }
  \newcommand{\xxxParagraphStar}[1]{\oldparagraph*{#1}\mbox{}}
  \newcommand{\xxxParagraphNoStar}[1]{\oldparagraph{#1}\mbox{}}
\fi
\ifx\subparagraph\undefined\else
  \let\oldsubparagraph\subparagraph
  \renewcommand{\subparagraph}{
    \@ifstar
      \xxxSubParagraphStar
      \xxxSubParagraphNoStar
  }
  \newcommand{\xxxSubParagraphStar}[1]{\oldsubparagraph*{#1}\mbox{}}
  \newcommand{\xxxSubParagraphNoStar}[1]{\oldsubparagraph{#1}\mbox{}}
\fi
\makeatother


\usepackage{longtable,booktabs,array}
\usepackage{calc} % for calculating minipage widths
% Correct order of tables after \paragraph or \subparagraph
\usepackage{etoolbox}
\makeatletter
\patchcmd\longtable{\par}{\if@noskipsec\mbox{}\fi\par}{}{}
\makeatother
% Allow footnotes in longtable head/foot
\IfFileExists{footnotehyper.sty}{\usepackage{footnotehyper}}{\usepackage{footnote}}
\makesavenoteenv{longtable}
\usepackage{graphicx}
\makeatletter
\newsavebox\pandoc@box
\newcommand*\pandocbounded[1]{% scales image to fit in text height/width
  \sbox\pandoc@box{#1}%
  \Gscale@div\@tempa{\textheight}{\dimexpr\ht\pandoc@box+\dp\pandoc@box\relax}%
  \Gscale@div\@tempb{\linewidth}{\wd\pandoc@box}%
  \ifdim\@tempb\p@<\@tempa\p@\let\@tempa\@tempb\fi% select the smaller of both
  \ifdim\@tempa\p@<\p@\scalebox{\@tempa}{\usebox\pandoc@box}%
  \else\usebox{\pandoc@box}%
  \fi%
}
% Set default figure placement to htbp
\def\fps@figure{htbp}
\makeatother





\setlength{\emergencystretch}{3em} % prevent overfull lines

\providecommand{\tightlist}{%
  \setlength{\itemsep}{0pt}\setlength{\parskip}{0pt}}



 
\usepackage[]{natbib}
\bibliographystyle{elsarticle-harv}


\usepackage{xcolor}
\newcommand{\cw}[1]{\textcolor{orange}{#1}}
\newcommand{\Junyi}[1]{\textcolor{purple}{#1}}
\newcommand{\Ruth}[1]{\textcolor{teal}{#1}}
\newcommand{\Greg}[1]{\textcolor{blue}{#1}}
\newcommand{\Yongqi}[1]{\textcolor{violet}{#1}}
\newcommand{\Linda}[1]{\textcolor{pink}{#1}}
\renewcommand\thefootnote{\textcolor{orange}{\arabic{footnote}}}
\newcommand{\cwfn}[1]{\footnote{\textcolor{orange}{CW: #1}}}
\newcommand{\Junyifn}[1]{\footnote{\textcolor{purple}{Junyi: #1}}}
\makeatletter
\@ifpackageloaded{caption}{}{\usepackage{caption}}
\AtBeginDocument{%
\ifdefined\contentsname
  \renewcommand*\contentsname{Table of contents}
\else
  \newcommand\contentsname{Table of contents}
\fi
\ifdefined\listfigurename
  \renewcommand*\listfigurename{List of Figures}
\else
  \newcommand\listfigurename{List of Figures}
\fi
\ifdefined\listtablename
  \renewcommand*\listtablename{List of Tables}
\else
  \newcommand\listtablename{List of Tables}
\fi
\ifdefined\figurename
  \renewcommand*\figurename{Figure}
\else
  \newcommand\figurename{Figure}
\fi
\ifdefined\tablename
  \renewcommand*\tablename{Table}
\else
  \newcommand\tablename{Table}
\fi
}
\@ifpackageloaded{float}{}{\usepackage{float}}
\floatstyle{ruled}
\@ifundefined{c@chapter}{\newfloat{codelisting}{h}{lop}}{\newfloat{codelisting}{h}{lop}[chapter]}
\floatname{codelisting}{Listing}
\newcommand*\listoflistings{\listof{codelisting}{List of Listings}}
\makeatother
\makeatletter
\makeatother
\makeatletter
\@ifpackageloaded{caption}{}{\usepackage{caption}}
\@ifpackageloaded{subcaption}{}{\usepackage{subcaption}}
\makeatother
\journal{Transportation Research Part C: Emerging Technologies}
\usepackage{bookmark}
\IfFileExists{xurl.sty}{\usepackage{xurl}}{} % add URL line breaks if available
\urlstyle{same}
\hypersetup{
  pdftitle={Towards Accelerating Transportation Research: Measuring the Practice of Open Science},
  pdfauthor={Junyi Ji; Ruth Lu; Yongqin Dong; Liming Wang; Bahman Madadi; Silvia Varotto; Nicolas Saunier; Gregory S. Macfarlane; Mostafa Ameli; Cathy Wu},
  pdfkeywords={Open Science, Large Language Models},
  colorlinks=true,
  linkcolor={blue},
  filecolor={Maroon},
  citecolor={Blue},
  urlcolor={Blue},
  pdfcreator={LaTeX via pandoc}}


\setlength{\parindent}{6pt}
\begin{document}

\begin{frontmatter}
\title{Towards Accelerating Transportation Research: Measuring the
Practice of Open Science}
\author[]{Junyi Ji%
%
}

\author[]{Ruth Lu%
%
}

\author[]{Yongqin Dong%
%
}

\author[]{Liming Wang%
%
}

\author[]{Bahman Madadi%
%
}

\author[]{Silvia Varotto%
%
}

\author[]{Nicolas Saunier%
%
}

\author[1]{Gregory S. Macfarlane%
%
}
 \ead{gregmacfarlane@byu.edu} 
\author[]{Mostafa Ameli%
%
}

\author[2]{Cathy Wu%
\corref{cor1}%
}
 \ead{cathywu@mit.edu} 

\affiliation[1]{organization={Brigham Young University, Civil and
Construction Engineering},addressline={430 Engineering
Building},city={Provo},postcode={84604},postcodesep={}}
\affiliation[2]{organization={Massachusetts Institute of
Technology, Civil and Environmental Engineering},,postcodesep={}}

\cortext[cor1]{Corresponding author}










        
\begin{abstract}
This is the abstract. Lorem ipsum dolor sit amet, consectetur adipiscing
elit. Vestibulum augue turpis, dictum non malesuada a, volutpat eget
velit. Nam placerat turpis purus, eu tristique ex tincidunt et. Mauris
sed augue eget turpis ultrices tincidunt. Sed et mi in leo porta
egestas. Aliquam non laoreet velit. Nunc quis ex vitae eros aliquet
auctor nec ac libero. Duis laoreet sapien eu mi luctus, in bibendum leo
molestie. Sed hendrerit diam diam, ac dapibus nisl volutpat vitae.
Aliquam bibendum varius libero, eu efficitur justo rutrum at. Sed at
tempus elit.
\end{abstract}





\begin{keyword}
    Open Science \sep 
    Large Language Models
\end{keyword}
\end{frontmatter}
    

\section{Introduction}\label{intro}

Open science initiatives have accelerated research progress across many
disciplines, including computational
science\textasciitilde{}\cite{peng2011reproducible},
psychology\textasciitilde{}\cite{open2015estimating}, and
statistics\textasciitilde{}\cite{stodden2014reproducible}. Recently,
open science practices have gained increasing recognition in the
transportation research community, as reflected by the introduction of
data and code availability statements in Transportation Research
journals\textasciitilde{}\cite{trc_guide2025}. Notably,
\textit{Transportation Research Part C} explicitly emphasizes open
science in its aims and scope, stating that ``Special emphasis is given
in open science initiatives and promoting the opening of large-scale
datasets that can support transferability and benchmarking of different
approaches\textasciitilde{}\cite{trc_aimsscope2025}.'\,' However,
despite these policy advancements, the extent to which open science
practices have been adopted and implemented in transportation research
remains largely unexamined. This motivates a systematic measurement of
open science practices in transportation research, with a particular
focus on the availability of data and code in papers published in
Transportation Research
journals.\cwfn{Maybe worth including in the motivation: Doing this measurement In a way that is repeatable (not a once-and-done study), or done at scale (many papers) (But why.) \Ruth{Added a sentence about this, I put down a possible reasoning as well.}}
Our goal is to understand current practices, identify commonly used
tools, and highlight barriers that may hinder broader adoption of open
science within the community. We aim to do this in a way that can be
repeated at scale to track progress in the field over time, potentially
helping to identify persisting issues preventing the sharing of code and
data.

The idea of open science \cite{woelfle2011open} is to make scientific
knowledge openly available, accessible, and reusable, as defined by the
United Nations Educational, Scientific and Cultural Organization
(UNESCO)\textasciitilde{}\cite{unesco_openscience2022}. It further
connects to the idea of reproducibility and replicability in science
\cite{national2019reproducibility}, where reproducibility refer to the
ability to obtain the same results using the same data and methods,
while replicability refers to the ability to obtain similar results
using different data or methods. Making data and code open is the first
and essential step towards achieving reproducibility and replicability,
as it allows other researchers to verify and build upon the original
work. Thus the primary focus of this article is on availability.

\%
\Junyi{[Junyi: I want to highlight the challenge and the opporunity for transportation research here.]}
Unlike many other fields, transportation research faces unique
challenges, which make availability a non-trivial task for the
community. Challenges from the practice side \%
\cwfn{Add privacy? Large datasets $\rightarrow$ Data storage challenges? \Ruth{good suggestions, should I try to find sources for these?}}
include privacy concerns, the size of datasets, the inherent complexity
and variety of transportation
systems\textasciitilde{}\cite{wu_curseVariety2023}, the heterogeneity of
data sources\textasciitilde{}\cite{welch2019big}, the diversity of
regional characteristics\textasciitilde{}\cite{sun2020identifying}, and
the need for active collaboration with a wide range of
stakeholders\textasciitilde{}\cite{nie2025brief}, such as government
agencies, private companies, and the public.

From a research perspective, transportation research is highly
interdisciplinary\textasciitilde{}\cite{sun2017discovering}, spanning
fields such as physics, economics, psychology, computational science,
environmental science, and more. This interdisciplinary nature also
means that a single problem can be approached from multiple disciplinary
perspectives, each bringing unique technical skills. Making data and
code openly available enables researchers from different backgrounds to
reuse resources, avoid duplicating effort, and focus on advancing
solutions. Ultimately, this accelerates research progress and fosters
greater collaboration across the community and an open research culture.

Measuring the state of open science practices is itself a complex and
resource-intensive
endeavor\textasciitilde{}\cite{stagge2019assessing,yang2020estimating,youyou2023discipline}.
Traditionally, this involves manually reviewing each paper to assess the
availability of data and
code\textasciitilde{}\cite{stagge2019assessing}, as well as verifying
the validity of provided resources. Such manual processes are
labor-intensive, time-consuming, and do not scale efficiently to large
corpora of research articles. Moreover, simply reporting the proportion
of papers with available data or code offers only a partial view.
Important contextual information - such as reasons for non-availability,
types of data and code used, and the tools or platforms employed - is
often not systematically captured. Without these details, it is
difficult to fully understand the barriers and opportunities for
advancing open science practices in transportation research.

To tackle these challenges, this article examines the state of open
science practices in transportation research by mining the full-text of
the publications in Transportation Research journals and leveraging
\cw{both bibliometric approaches and} Large Language Models (LLMs) to
automatically extract features related to data and code availability. We
focus on the following research questions:


\bibliography{main.bib}



\end{document}
